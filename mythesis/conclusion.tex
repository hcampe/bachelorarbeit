% !TEX root = mythesis.tex

%==============================================================================
\chapter{Fazit}
\label{sec:fazit}
%==============================================================================

Um den Ansatz für die vorgestellte Demonstration von Confinement im Rahmen der
SU(2)-Eichsymmetrie nachvollziehen zu können, waren zunächst die theoretischen
Hintergründe notwendig. Hierbei wurde als Erstes der Metropolis-Algorithmus
vorgestellt, welcher das fundamentale Werkzeug zur numerischen Approximation
beliebiger Feynman-Pfadintegrale darstellt. Damit wurde als Nächstes exemplarisch
der harmonische Oszillator vorgestellt, hier konnten Grundzustandsenergie und
-wellenfunktion erfolgreich mit Simulationen angenähert werden.

Nun folgte die Betrachtung des von Wilson vorgelegten gitterfeldtheoretischen
Pendants zur Yang-Mills-Wirkung, welches die Dynamik des Eichfelds der
SU(2)-Eichsymmetrie auf einem Gitter beschreibt. Dies ermöglichte dann die
Betrachtung dieser Dynamik mit Hilfe des Metropolisalgorithmus sowie der
Definition der Wilson-Loops als Observable zur Bestimmung des statischen
Potentials eines Quark-Antiquark-Paares.

Die Beschränkung auf SU(2) hatte den Vorteil, dass sich die Eichtransformationen
und die gitterfeldtheoretische Formulierung des Eichfeldes in Form von
Paralleltransporten relativ leicht als unitäre $2 \times 2$-Matrizen mit
Determinante 1 numerisch darstellen ließen. Die Implementierung erfolgte dann
mit C++, es wurden Simulationen auf einem Gitter mit $10^4$ Gitterpunkten
zur Vermessung von Wilson-Loops verschiedener Maße vorgenommen.

Schließlich konnte im Rahmen der Auswertung das statische Potential aus den
gewonnenen Daten extrahiert werden. Der daran vorgenommene Fit zeigte, dass
die ausgeführten Messungen noch verbesserungswürdig sind. Trotzdem konnte
deutlich ebenjener linearer Term im statischen Potential wiedergefunden werden,
den Wilson (wie eingangs erwähnt) ursprünglich vorhergesagt hatte. So konnte
numerisch das Auftreten von Confinement im Rahmen der SU(2)-Eichsymmetrie
nachvollzogen werden.

Leider ist diese Erkenntnis für die physikalische Realität nicht direkt
relevant, da nicht die SU(3)-Eichsymmetrie betrachtet wurde. Die vorgestellten
Konzepte ließen sich aber auch dieses Problem übertragen, hierfür wäre vorallem
eine numerische Repräsentation von Elementen dieser Gruppe notwendig. Alternativ
bietet sich für eine weitergehende Betrachtung die genauere Untersuchung des
Phasenübergangs zwischen Confinement und Deconfinement an.

%Leider ist diese Erkenntnis für die physikalische Realität nur insofern
%bedingt relevant, alsdass die Energieskala, bei der Confinement auftritt weit
%unter der liegt, ab der die elektroschwache Vereinheitlichung ihren
%Gültigkeitsbereich hat, welche von SU(2)$\times$U(1) generiert wird.
%Als nächster Schritt wäre deshalb eine Betrachtung der SU(3)-Eichsymmetrie
%naheliegend, um tatsächlich Confinement für die starke Wechselwirkung zu
%demonstrieren.
