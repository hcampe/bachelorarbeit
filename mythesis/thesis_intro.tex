% !TEX root = mythesis.tex

%==============================================================================
\chapter{Einleitung}
\label{sec:einleitung}
%==============================================================================

Die theoretische Teilchenphysik erfuhr eine Reihe von Durchbrüchen, nachdem
Gell-Mann und Zweig 1964 das Konzept von Quarks eingeführt hatten. So benannten
sie die \enquote{Bausteine} des in den vorherigen Jahren immer größer gewordenen
Teilchenzoos und erklärten insbes. die von Gell-Mann und Ne'eman zuerst beschriebenen
SU(3)-Multiplets von Baryonen~\cite{historyOfQCD}. 1971 führten Gell-Mann und Fritzsch
die Farbladung der Quarks als weitere Quantenzahl ein, um Zustände aus drei
gleichen Quarks mit dem Pauli-Prinzip in Einklang zu bringen. Im darauffolgenden
Jahr betrachteten sie dann die Farb-Gruppe SU(3) als Eichgruppe und legten so
die Grundsteine einer neue Theorie, der Quantenchromodynamik~\cite{historyOfQCD}.
Eine wichtige Eigenschaft dieser neuen Theorie ist das sog. \emph{Confinement}:
Quarks können nie ungebunden auftreten. Rein physikalisch lässt
sich diese Eigenschaft darauf zurückführen, dass Gluonen (die Mediatoren der 
beschriebenen, sog. starken Wechselwirkung) selbst Farbladungen tragen, also anders
als z.\,B. Photonen auch untereinander interagieren können~\cite{latticeQCDhistory}.

Leider hat die Quantenchromodynamik einen beträchtlichen Nachteil gegenüber
der Quantenelektrodynamik: Bei niedrigen Energien lässt sie sich nicht perturbativ
betrachten. Hierdurch sind analytische Betrachtungen in diesem Bereich sehr schwierig,
weswegen andere Methoden vonnöten sind. Eine solche Alternative bietet eine
gitterfeldtheoretische Formulierung, welche erstmals 1974 von Wilson vorgelegt
wurde~\cite{latticeQCDhistory}. Dabei präsentierte er gleichzeitig eine erste
Motivation für Confinement, welche fünf Jahre später numerisch bestätigt werden konnte.

In dieser Arbeit soll diese numerische Betrachtung nachvollzogen
werden: Der Einfachheit halber wird hier nur die SU(2)-Eichsymmetrie betrachtet:
Die Simulation des dazugehörigen Eichfelds bietet die Möglichkeit, den Wert des
statischen Potentials eines Quark\footnote{Am treffendsten wäre wohl die Bezeichnung
als \enquote{Pseudoquark}, da es sich nicht um die SU(3)-Eichsymmetrie
handelt.}-Antiquark-Paars für
verschiedene Abstände zu berechnen. Confinement sollte dann dadurch zu erkennen
sein, dass das Potential (neben einem Coulomb-artigen Verhalten für kleine Abstände)
für große Abstände \emph{linear} ansteigt. Entsprechend bleibt die Kraft zwischen
Quark und Antiquark konstant, wodurch man das Paar nicht voneinander isolieren kann.

Folgender Aufbau wurde für die folgenden Abschnitte gewählt: In Kapitel
\ref{sec:theorie} wird mit dem Metropolisalgorithmus erst die allgemeine
numerische Methode
vorgestellt, welche die numerische Approximation von Feynmans Pfadintegralen
ermöglicht. Darauf folgt dann die theoretische Anwendung auf die SU(2)-Eichsymmetrie
-- von einer allgemeinen Formulierung der Wirkung zur konkreten Observable für die 
Bestimmung des statischen Potentials. Kapitel \ref{sec:implementation}
enthält dann die Details der konkreten Implementation und den Ablauf der 
Simulation und in Kapitel \ref{sec:auswertung} werden die gewonnenen
Daten ausgewertet und die Ergebnisse präsentiert.

